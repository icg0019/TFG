\capitulo{2}{Objetivos del proyecto}

Este apartado explica de forma precisa y concisa cuales son los objetivos que se persiguen con la realización del proyecto. Se puede distinguir entre los objetivos marcados por los requisitos del software a construir y los objetivos de carácter técnico que plantea a la hora de llevar a la práctica el proyecto.


\section{Objetivos generales}
El objetivo del proyecto es realizar una interfaz que permita a los usuarios mejorar la forma en la que realizan sus pedidos de briks. Para ello, se realizará lo siguiente: 

\begin{itemize}

	\item Desarrollar un algoritmo para optimar los pedidos semanales de brick para evitar al máximo las penalizaciones teniendo en cuenta los distintos criterios de penalizaciones. 
	
	\item Desarrollar una interfaz para que los usuarios puedan realizar lo siguiente: 
\begin{itemize}

	\item Loguearse con su usuario

	\item Visualizar los escenarios óptimos de los pedidos, dando a elegir al usuario si lo quiere por coste de penalización por pedido o por coste de penalización por brik. 

	\item Visualizar los pedidos hechos anteriormente. 
	
	\item Visualizar el stock actual de cada una de las referencias de briks

\end{itemize}
\end{itemize}
\section{Objetivos técnicos}

\begin{itemize}

\item Utilizar Python como lenguaje para el algoritmo de optimización

\item Crear un almacén de datos con diferentes orígenes (SAP, base de datos) con Data Fabric con el fin de dejar preparados los datos para su utilización 

\item Utilizar Python y Django para la visualización web

\item Usar Scrum como metodología de planificación ágil

\item Utilizar Git como sistema de control de versiones junto con la plataforma GitHub.

\end{itemize}

\section{Objetivos personales}
\begin{itemize}
\item Evitar al máximo las pérdidas económicas por las penalizaciones en los pedidos semanales 
\item Facilitar el trabajo al departamento de aprovisionamiento de la compañía
\item Aprender el lenguaje Python para las futuras mejoras de procesos
\item Utilizar librerías de Python que no conocía y me ayudarán en las siguientes mejoras de procesos. 
\end{itemize}

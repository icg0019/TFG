\capitulo{1}{Introducción}


Frías Nutrición S.A.U es una empresa dedicada a la fabricación de productos saludables de origen vegetal, destacando en sus bebidas y caldos. En los últimos años, ha duplicado su facturación, lo que está ocasionando un rápido aumento de la demanda de datos y revisión de los procesos. 


Para la fabricación de las bebidas y caldos, se hacen pedidos de briks semanalmente en función de las necesidades de producción que haya y del stock de seguridad que se quiera de cada una de las referencia, con la particularidad de que dependiendo como se programe este pedido, se aplicarán unas penalizaciones o no. Las penalizaciones van en función de la fabricación, los tramos utilizados, los colores adicionales y el número de briks (se explicarán los criterios específicos más adelante). El último año ha habido una gran cantidad de penalizaciones por no hacer los pedidos de la forma más óptima, por lo que se ve una clara necesidad de buscar una solución. 


En este proyecto se propone realizar un algoritmo que calcule la forma óptima de realizar los pedidos en función de los briks que haya que comprar, basándose en los criterios para evitar penalizaciones y en el valor mínimo y máximo de cada referencia de brik que se necesite. Los valores de máximo y mínimo de cada referencia se cogerán de un almacén de datos, previo cálculo y preparación de dichos datos. 
